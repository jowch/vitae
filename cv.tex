\documentclass[line,margin]{res}
\usepackage{pifont}
\usepackage{tabularx}
\usepackage{multicol}
\usepackage{hyperref}


\oddsidemargin -.5in
\evensidemargin 0in
\textwidth=6.2in
\textheight=10in
\itemsep=0in
\parsep=0in
% if using pdflatex:
% \setlength{\pdfpagewidth}{\paperwidth}
% \setlength{\pdfpageheight}{\paperheight}

\newenvironment{list1}{
  \begin{list}{}{%
      \setlength{\itemsep}{0in}
      \setlength{\parsep}{0in} \setlength{\parskip}{0in}
      \setlength{\topsep}{0in} \setlength{\partopsep}{0in}
      \setlength{\leftmargin}{0.17in}}}{\end{list}}
\newenvironment{list2}{
  \begin{list}{\boldmath$\cdot$}{%
      \setlength{\itemsep}{0in}
      \setlength{\parsep}{0in} \setlength{\parskip}{0in}
      \setlength{\topsep}{0in} \setlength{\partopsep}{0in}
      \setlength{\leftmargin}{0.2in}}}{\end{list}}

% disable page numbers
\pagenumbering{gobble}

\begin{document}

\vspace*{-0.5in}
\name{Jonathan W. Chen \vspace*{.1in}}

\begin{resume}
\section{\sc Contact Information}
\vspace{.05in}
\begin{tabularx}{\textwidth}{@{} X @{} X @{}}
Ph.D. Student, Wong Lab & {\it E-mail:} \href{mailto:jwhc@ucla.edu}{jwhc@ucla.edu} \\
Department of Bioengineering & {\it Social:}
\href{https://jowch.github.io}{Website}
\href{https://www.linkedin.com/in/jonathan-chen-54648478/}{LinkedIn}
\href{https://github.com/jowch}{GitHub} \\
University of California, Los Angeles
\end{tabularx}


\section{\sc Research Interests}
Scientific applications of machine learning, human-in-the-loop learning,
computer-assisted decision \\ making, Bayesian methods, and
active learning in medicine and biological systems.


\section{\sc Education}
{\bf University of California, Los Angeles}, Los Angeles, California, USA \vspace{0.3em} \\
Ph.D. Student, Bioengineering. Advisor: Professor Gerard C. L. Wong, PhD.

{\bf Washington University in St. Louis}, St. Louis, Missouri USA \vspace{0.3em} \\
B.S. and M.S. Candidate, Computer Science with second major in Genomics
and Computational Biology and minor in Bioinformatics. May 2020.
\vspace*{0.3em}
\begin{list2}
\vspace*{.05in}
\item Thesis Topic: ``Rapid Generalized Psychometric Function Estimation
with Active Learning''
\item Advisor: Professor Dennis Barbour, MD, PhD.
\end{list2}


\section{\sc Awards and Honors}
{\bf Academic} \vspace{0.3em} \\
Dean's List, Spring 2018

{\bf Teaching} \vspace{0.3em} \\
Computer Science and Engineering: Outstanding Senior Award, Spring 2019



\section{\sc Academic Experience}
{\bf Washington University in St. Louis}, St. Louis, Missouri USA \vspace{-0.7em}

{\em Graduate Student} \hfill {\bf June 2018 - June 2020} \\
Laboratory of Sensory Neuroscience and Neuroengineering {\it PI:} Dennis Barbour \\
\vspace*{-.1in}
\begin{list2}
\item Led a project to develop active multi-dimensional, general
psychometric function estimation with Gaussian Processes and Bayesian active
learning. This includes the generalization of error function likelihood for
guess and lapse rate classification problems and the extension of an existing
uncertainty estimation function.
\end{list2}

{\em Head Teaching Assistant} \hfill {\bf August 2018 - June 2020} \vspace{0.3em} \\
(1) Introduction to Data Science (2) Cloud Computing and Big Data \\
\vspace*{-.1in}
\begin{list2}
\item Sharing administrative responsibilities with faculty instructor,
fielding of student inquiries ($120+$ students), and coordinating a team
of 10-15 undergraduate student teaching assistants and graders.
\item Proposing, creating, and updating homework and in-class lab assignments. Grading of homework and exams.
\end{list2}

{\em Student Researcher} \hfill {\bf January 2017 - December 2017}\\
Ding Lab \\
\vspace*{-.1in}
\begin{list2}
\item Mined 11,000 cancer patient tumor genomes to study rare copy number
variants and mutagenesis mechanisms.
\item Automated and scaled analysis pipeline to be run either locally or on
an IBM LSF computing cluster.
\end{list2}

{\em Teaching Assistant} \hfill {\bf January 2017 - May 2018}\\
Computer Science II \\
\vspace*{-.1in}
\begin{list2}
\item Assisted with lab studios and answered questions during office hours. Also helped grading of homework and exams.
\end{list2}

\newpage 
\section{\sc Professional Experience}
{\bf IBM}, San Jose, California USA

\vspace{-0.3cm}
{\em Senior Software Engineer, Intern} \hfill {\bf June 2019 - August 2019}\\
\vspace*{-.1in}
\begin{list2}
\item Undertook user-facing web application development for a
one-stop-shop, AI pipeline component sharing platform.
\item Collaborated closely with AI and backend engineers to implement a
user-interface exposing Kubeflow backend with React.
\end{list2}

{\bf Stentor Technology}, Walnut Creek, California USA

\vspace{-0.3cm}
{\em Full Stack Developer} \hfill {\bf June 2017 - December 2017}\\
\vspace*{-.1in}
\begin{list2}
\item Led the development and deployment of a token value exchange system
via Solidity contract for Ethereum blockchain.
\item Designed client-facing mobile app for customers to manage their token
assets, including authentication, marketplaces, and transfer interfaces using
React Native.
\item Engineered robust REST APIs using Node.js and Express.js to interface
between Transact-SQL database and front-end interfaces.
\end{list2}


% \section{\sc Papers in preparation}


\section{\sc Publications}

% \textbf{2019-2021} \\ \vspace{-0.5em}

\begin{list}{*}{}

\item[3.] Calvin K. Lee, William C. Schmidt, Shanice S. Webster, \textbf{Jonathan W. Chen}, George A. O'Toole, and Gerard C. L. Wong.
Broadcasting of amplitude- and frequency-modulated c-di-GMP signals facilitates
cooperative surface commitment in bacterial lineages. \textit{PNAS}. 2022

\item[2.] \textbf{J.W. Chen}, T. Larsen, M. Neumann.
Exploring Unfairness and Bias in Data. \textit{Proceedings of the AAAI
Conference on Artificial Intelligence}. 2020.

\item[1.] M. Neumann, \textbf{J.W. Chen}. Introduction to Python for Data Science.
\textit{Proceedings of the AAAI Conference on Artificial Intelligence}. 2019.

\end{list}

% \section{\sc Conference Presentations}

% \textbf{J.W. Chen}, T. Larsen, M. Neumann. 2020. Exploring Unfairness and
% Bias in Data. \textit{10th Symposium on Educational Advances in Artificial
% Intelligence, New York, USA, February, 2020}. Oral Presentation.

% J.W. Chen, M. Neumann. 2019. Introduction to Python for Data Science. 9th
% Symposium on Educational Advances in Artificial Intelligence, Hawaii, USA,
% January, 2019. Oral presentation.

\end{resume}
\end{document}
